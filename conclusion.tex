\chapter{結論}\label{chap:conclusion}

%本研究では, VR環境における視覚情報の変容が, 人間の食欲および味覚体験に与える影響を明らかにすることを目的とし, 通常視界とネガポジ反転視界を用いた比較実験を行った. 得られた知見を以下に総括する. 

%第一に, 視覚情報の色彩的な変容は, 食欲の減退に決定的な影響を与えることが確認された. アンケート結果において, 「見た目がおいしそうだった」という評価および「食べたい気持ちの持続性」がネガポジ反転条件下で著しく低下したことは, 人間が食品を認識する際, その色彩から得られる「おいしさの予測」に強く依存していることを示している. 

%第二に, 視覚情報は味覚の知覚プロセスにも干渉し, 喫食体験全体の質を左右することが明らかとなった. 同一の食品を摂取しているにもかかわらず, ネガポジ反転条件下では「味を通常通り感じた」とする評価が有意に低かった. これは, 視覚と味覚のクロスモーダルな相互作用により, 不自然な視覚入力が味覚の正常な処理を阻害し, 満足度を低下させる要因となったことを示唆している. 

%以上のことから, VR技術を用いた食環境の提示においては, 解像度や没入感のみならず, 色彩情報の正確性や自然さが喫食体験の質を維持する上で極めて重要であると結論付けられる. 

%本研究の成果は, 視覚情報を意図的に操作することで食欲を制御する「デジタルダイエット」や, 逆に食欲を増進させる食育支援システム等の開発に向けた基礎的な知見を提供するものである. 今後は, 被験者数の拡大や異なる視覚エフェクト(彩度の変化や特定の色の強調など)を用いた検証を重ねることで, 視覚と食欲の相関メカニズムをより詳細に解明していく必要がある. 


