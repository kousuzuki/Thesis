\chapter{実験}\label{chap:experiment}

参加者には、1杯のカップラーメンを半分ずつに分け、VRゴーグルを装着した状態で、通常視界条件とネガポジ反転条件の2条件でそれぞれ半分を摂取してもらった。
参加者は事前アンケート(空腹度、見た目による食欲)を記入した。

カップラーメンは市販の同一製品を使用し、実験者によりお湯を注いで3分待機した後、麺・具・スープをキッチンスケールを用いて正確に半分ずつ2つのどんぶりに分けた。これにより、味や温度の条件を両条件間で均等化した。
参加者をランダムに2グループ(A群・B群)に割り当て、順序効果を打ち消した。
•  A群:1回目に通常視界 → 2回目にネガポジ反転視界
•  B群:1回目にネガポジ反転視界 → 2回目に通常視界

実験の流れは以下の通りである。
 参加者にVRゴーグルを装着させ、視界の調整と簡単な練習を行った。
 1回目の半分を指定された視界条件で摂取させた(制限時間なし、自然な速度で)。
 摂取終了直後に、1回目専用のアンケート(見た目の魅力、食欲、味の知覚、不快感、満足度など、全6項目)を記入させた。
 VRゴーグルの視界設定を切り替え、すぐに2回目の残り半分を別の視界条件で摂取させた。
 2回目摂取終了直後に、同一内容のアンケートを記入させた。
 最後に総合的な自由記述質問(どちらの視界が食べにくかったか、再挑戦の意向など)を記入させた。
 VRゴーグルを外し、休憩を取った後、必要に応じて口頭での感想を聴取し実験を終了した。

