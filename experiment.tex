\chapter{実験}\label{chap:experiment}

本章では, VR視界下における視覚情報の変容が食欲および味覚に与える影響を検証するため, 実施した実験の方法および結果について述べる. 

\section{実験方法}

\subsection{被験者}
被験者は20歳から26歳の男性10名とした. 全ての被験者に対し, 実験の目的と安全性を十分に説明し, 同意を得た上で実施した. 

\subsection{実験機材および環境}
実験にはVRヘッドマウントディスプレイ(HMD)である「Meta Quest 3」を使用した. パススルー機能を用い, 外部カメラの映像をリアルタイムで処理することで, 「通常視界」および「ネガポジ反転視界」を提示した. 
また, 実験に使用する食品として, 市販の同一製品のカップラーメンを使用した. 味や温度の条件を均等化するため, 実験者がお湯を注いで3分待機した後, 麺・具を半分ずつ摂取した. 

\subsection{実験デザイン}
本実験では, 提示順序による順序効果(慣れや満腹感の影響)を相殺するため, 被験者をランダムに以下の2群に割り当てた. 

\begin{itemize}
    \item \textbf{A群}:1回目に「通常視界」, 2回目に「ネガポジ反転視界」の順で摂取
    \item \textbf{B群}:1回目に「ネガポジ反転視界」, 2回目に「通常視界」の順で摂取
\end{itemize}

\subsection{手順}
実験は以下の手順で進行した. 

\begin{enumerate}
    \item \textbf{事前測定}:被験者に現在の空腹度および見た目による食欲についての事前アンケートを記入させた. 
    \item \textbf{機材装着}:VRゴーグルを装着させ, 視界の調整および操作の練習を行った. 
    \subsection{試食セッション(1回目)}:指定された視界条件下で, 用意されたカップラーメンの半分を摂取させた. 摂取時間に制限は設けず, 自然な速度で完食させた. 
    \item \textbf{中間アンケート}:摂取終了直後に, 見た目の魅力, 食欲, 味の知覚, 不快感, 満足度等の6項目(7段階リッカート尺度)について回答させた. 
    \item \textbf{試食セッション(2回目)}:視界設定を切り替え, 残りの半分を別の視界条件下で同様に摂取させた. 
    \item \textbf{事後アンケート}:2回目の摂取終了後, 1回目と同様のアンケートに加え, 総合的な自由記述(食べにくさの比較, 再挑戦の意向等)および口頭でのインタビューを行い, 実験を終了した. 
\end{enumerate}

\begin{figure}[H]
\begin{center}
\includegraphics[width=0.3\linewidth]{figs/IMG_5322.jpg}
\caption{食事の様子}
\label{fig:exp_human}
\end{center}
\end{figure}

\begin{figure}[H]
\begin{center}
\includegraphics[width=0.3\linewidth]{figs/cupnoodle.jpg}
\caption{ネガポジ反転されたカップラーメン}
\label{fig:exp_cupnoodle}
\end{center}
\end{figure}




\section{結果}

本節では, 収集したアンケートデータを統計的に解析した結果を報告する. 各条件下における回答の平均値, 中央値, 最頻値, および分散を算出し, 視覚情報の変容(ネガポジ反転)が食事体験に与える影響を比較した. 

\subsection{通常時とネガポジ視界時の比較}

表\ref{tab:normal_results}および表\ref{tab:negaposi_results}に, 各条件下における集計値を示す.  @@@$n=10$で最頻値はおかしいです@@@

% --- 表1: 通常時 ---
\begin{table}[H]
  \centering
  \caption{通常時の視界におけるアンケート結果($n=10$)}
  \label{tab:normal_results}
  \begin{tabular}{lcccc}
    \toprule
    項目 & 平均値 & 中央値 & 最頻値 & 分散 \\
    \midrule
    お腹の空き具合 & 4.8 & 5.0 & 5 & 1.56 \\
    カップラーメンを食べたい気分 & 4.9 & 4.5 & 4 & 1.09 \\
    見た目がおいしそうだった & 5.3 & 6.0 & 6 & 2.61 \\
    味を通常通り感じたか & 6.2 & 7.0 & 7 & 1.56 \\
    食べたい気持ちは続いたか & 6.2 & 7.0 & 7 & 1.36 \\
    視界の状態が不快だった & 3.3 & 3.0 & 1 & 3.81 \\
    全体の満足度 & 5.7 & 6.5 & 7 & 2.81 \\
    \bottomrule
  \end{tabular}
\end{table}

% --- 表2: ネガポジ時 ---
\begin{table}[H]
  \centering
  \caption{ネガポジ視界におけるアンケート結果($n=10$)}
  \label{tab:negaposi_results}
  \begin{tabular}{lcccc}
    \toprule
    項目 & 平均値 & 中央値 & 最頻値 & 分散 \\
    \midrule
    お腹の空き具合 & 4.7 & 5.0 & 3, 5, 6, 7 & 3.41 \\
    カップラーメンを食べたい気分 & 4.3 & 4.5 & 7 & 4.61 \\
    見た目がおいしそうだった & 1.7 & 1.0 & 1 & 1.41 \\
    味を通常通り感じたか & 3.6 & 3.0 & 3 & 3.24 \\
    食べたい気持ちは続いたか & 2.9 & 2.5 & 2 & 1.69 \\
    視界の状態が不快だった & 3.8 & 4.5 & 5 & 3.56 \\
    全体の満足度 & 3.0 & 3.0 & 3 & 0.80 \\
    \bottomrule
  \end{tabular}
\end{table}

\subsection{結果の要約}

アンケート結果の比較から, 以下の傾向が明らかとなった. 

\subsubsection{外観評価と食欲の持続性}
「見た目がおいしそうだった」という項目において, 通常時の平均値が5.3であったのに対し, ネガポジ視界時では1.7と大幅に低下した. これに伴い, 「食べたい気持ちは続いたか」という項目も, 通常時の6.2からネガポジ時の2.9へと大きく減少している. このことから, 視覚情報のネガポジ化は, 対象物の食欲を著しく減退させ, 食意欲の維持を困難にさせることが示唆された. 

\subsubsection{味覚体験への影響}
「味を通常通り感じたか」という項目では, 通常時6.2に比べてネガポジ時3.6は低い評価となった. 食品自体は同一であるにもかかわらず, 視覚的な違和感が味覚の知覚や評価に対して負の影響を与えている可能性が示された. 

\subsubsection{不快感と全体的満足度}
視界の状態に対する不快感は, 通常時の3.3に対し, ネガポジ時では3.8とやや増した. また, 全体の満足度については通常時の5.7からネガポジ時の3.0へと半減しており, 視覚情報の不自然さが食事体験全体の質を大きく低下させる要因となっていることが確認された. 
