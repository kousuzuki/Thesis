\chapter{序論}

\section{研究背景}
 
近年,肥満や生活習慣病の増加が社会問題となる中で,過食を効果的に抑制する手法の一つとして,
食べ物の色彩操作が注目されている.食行動において視覚情報は味覚や嗅覚と同様に食欲へ強い影響を及ぼすことが知られており,
先行研究では,食品の色彩が「おいしさの評価」や「食欲の増減」に関与することが示されている.
一般に,暖色系は食欲を高め,寒色系は食欲を低下させる傾向が報告されている.
さらに,人は食品の色が普段のイメージから大きく外れている場合に食欲が減衰することも指摘されている.
しかし,過食を抑制することを目的として食品の色を物理的に調整することは,
着色料や照明の使用により質感・明るさといった色以外の要素まで変化してしまうため,
純粋に色のみの効果を評価する上で困難が伴う.
この課題に対し,AR(Augmented Reality)などの拡張現実技術を用いることで,
現実の食品に対して任意の色変換を重畳でき,食行動における視覚刺激の操作が容易に実現可能となる.
このように,HMDを用いた色彩操作は,過食抑制や食生活改善に向けた新たなアプローチとして有望視されている.
本研究では,「色の反転」という極端な視覚変換に着目し,その操作が食欲にどのような影響を及ぼすかを調査する

\begin{figure}[H]
\begin{center}
\includegraphics[width=0.3\linewidth]{figs/Noodle.jpg}
\caption{ネガポジ反転されたラーメン}
\label{fig:}
\end{center}
\end{figure}


\section{関連研究}

青色フィルタによる食欲を減衰させる研究では, シースルー型のHMD (Head Maunt Display)を用いて, 目の前にある料理に青色フィルタを重畳させる手法が提案されている. 
この方法では, 食事が進むにつれて青色フィルタの強度を段階的に強めることで, 料理を美味しく食べながらも視覚的に食欲を減衰させる効果を狙っている. 
常時青色フィルタを適用する手法やフィルタを適用しない手法と比較した結果, 
提案手法では食欲の低下, 満腹感の増加, そして食事時間の延長が確認されており, 食行動の制御に有効であることが示された. 



\ref{chap:purpose}章で目的を述べる. 

% dvipdfmxとhereのテスト
%\begin{figure}[H]
%	\begin{center}
%		\includegraphics[width=1.0\linewidth]{../zero.png}
%		\caption{}
%		\label{fig:}
%	\end{center}
%\end{figure}
%
