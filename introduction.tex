\chapter{序論}

\section{研究背景}
 
近年,肥満や生活習慣病の増加が社会問題となる中で,過食を効果的に抑制する手法の一つとして,食品の色彩の操作というアプローチがある. %\cite{schlintl2020}
食行動において視覚情報は, 味覚や嗅覚と同様に食欲へ強い影響を及ぼすことが知られている. 先行研究\cite{奥田ら2002}では, 食品の色彩が「食欲の増減」に直接関与することが示されており, 
暖色系は食欲を増進させ, 寒色系は食欲を減退させる傾向があることが報告されている. 

食品の色彩を変化させる手法として, AR(Augmented Reality:拡張現実)技術や, カメラを通した現実世界を映し出す装着型ディスプレイ型のHMD(Head Mounted Display)を用いる手法がある.
AR技術を用いれば, 現実の食品の質感を維持したまま, デジタル処理によって任意の色変換を重畳することが可能となる. 

村井\cite{村井ら2020}の研究では,シースルー型HMDを用いて食品に青色フィルタを重畳させることで,美味しく食べながらも満腹感を高め食事量を抑制する手法が提案されている.
しかし,特定の色を重畳する方法では,視覚的変化が色味の違いに限定され,食品本来の外観が維持されやすいためフィルタの効果が減退してしまう.

特定の色の重畳よりも劇的な視覚変容をもたらす手法として, ネガポジ反転(補色変換)がある. 
ネガポジ反転処理は,色味だけでなく明度や彩度を網羅的に逆転させるため,図1のような強い違和感を創出できる.

そこで本研究では, VRゴーグルを用いたリアルタイムのネガポジ反転処理が, 被験者の食欲および食品に対する主観的評価にどのような影響を及ぼすかを検証する. 



 

\begin{figure}[H]
\begin{center}
\includegraphics[width=0.3\linewidth]{figs/Noodle.jpg}
\caption{ネガポジ反転されたラーメン}
\label{fig:intro_Noodle}
\end{center}
\end{figure}


\section{関連研究}

青色フィルタによる食欲を減衰させる研究では, シースルー型のHMD (Head Maunt Display)を用いて, 目の前にある料理に青色フィルタを重畳させる手法が提案されている. 
この手法では, 食事が進むにつれて青色フィルタの強度を段階的に強めることで, 料理を美味しく食べながらも視覚的に食欲を減衰させる効果を狙っている. 
常時青色フィルタを適用する手法やフィルタを適用しない手法と比較した結果, 
提案手法では食欲の低下, 満腹感の増加, 食事時間の延長が確認されており, 食行動の制御に有効であることが示された. 
本研究では,「色の反転」という極端な視覚変換に着目し,その操作が食欲にどのような影響を及ぼすかを調査する.

\section{研究目的}
本研究ではネガポジ反転システムの開発, ネガポジ反転処理が食欲・味覚に与える影響の検証を目的とする. 

 

% dvipdfmxとhereのテスト
%\begin{figure}[H]
%	\begin{center}
%		\includegraphics[width=1.0\linewidth]{../zero.png}
%		\caption{}
%		\label{fig:}
%	\end{center}
%\end{figure}
%
