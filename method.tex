\chapter{提案手法}\label{chap:method}


VRヘッドマウントディスプレイ(HMD)のビデオシースルー機能(パススルー)を活用し, 現実世界の視覚情報をリアルタイムで加工・提示するシステムを構築した. 

\subsection{開発環境および使用機材}
本システムの開発および実験には, 表\ref{tab:development_env}に示す環境および機材を使用した. HMDには, 高解像度のカラーパススルー機能を備えたMeta Quest 3を採用した. 

\begin{table}[H]
  \centering
  \caption{開発環境および使用機材}
  \label{tab:development_env}
  \begin{tabular}{ll}
    \toprule
    項目 & 内容 \\
    \midrule
    HMD & Meta Quest 3 \\
    ゲームエンジン & Unity 2022.3.52f1 \\
    開発ツールキット & Meta OpenXR SDK / Passthrough camera API Samples / Meta All In One SDK \\
    ビルド環境 & Android OS (Quest Native) \\
    \bottomrule
  \end{tabular}
\end{table}

\subsection{視覚変容フィルターの実装}
視覚情報の変容を実現するため, Meta社が提供している公式サンプルプロジェクト「PassthroughCameraApiSamples」を基に, 独自のシェーダエフェクトを実装した. 

具体的には, サンプル内に含まれる「Shader Sample」をベースとして活用し, パススルー映像の指定範囲のピクセルに対して階調反転処理(ネガポジ反転)を行うカスタムシェーダを記述した. 
このシェーダでは, カメラから入力されたRGBの各輝度値 $I$ に対し, 出力値 $I'$ を $I' = 1.0 - I$ と定義することで, 色相と明度を反転させている. 


\subsection{アプリケーションのデプロイ}
実装したネガポジ反転フィルターは, Unityのビルド設定においてAndroidプラットフォーム向けに最適化を行い, apk形式のアプリケーションとしてMeta Quest 3にインストールした. 
これにより, PCとの有線接続による遅延を排除し, 被験者が頭部を自由に動かした際にも, 遅延などの違和感を最小限に抑えた状態で, パススルー映像を映すことが可能となった. 




%\begin{figure}[h]
       % \begin{center}
        %\includegraphics[width=1.0\linewidth]{figs/vq_map_128part.eps}
        %\caption{Representative Vectors of the $N_c = 128$ Map}
        %\label{fig:vq_map_128part}
       % \end{center}
%\end{figure}






